% Copyright (C) 2001 Python Software Foundation
%
% Use the following until this is added to Python's standard library
\documentclass{howto}
\title{mimelib -- A MIME and RFC822 handling package}
\release{0.2}
\author{Barry A. Warsaw}
\authoraddress{barry@digicool.com}
\begin{document}
\maketitle

%\section{\module{mimelib} -- A MIME and RFC822 handling package}

%\declaremodule{extension}{mimelib}
%\modulesynopsis{A package for parsing, writing, and managing MIME messages.}
%\moduleauthor{Barry A. Warsaw}{barry@digicool.com}
%\sectionauthor{Barry A. Warsaw}{barry@digicool.com}

The \module{mimelib} package is a library for managing MIME and other
\rfc{822}-based message documents.  It subsumes most of the functionality
in several older standard modules such as \module{rfc822},
\module{mimetools}, \module{multifile}, and other non-standard packages such
as \module{mimecntl}.

The primary distinguishing feature of \module{mimelib} is that it
splits the parsing and generating of MIME documents from the internal
\emph{object model} representation of MIME.  Applications
using \module{mimelib} deal primarily with objects; you can add
sub-objects to messages, remove sub-objects from messages, completely
re-arrange the contents, etc.  There is a separate parser class and a
separate generator class which handles the transformation from flat
text to the object module, and then back to flat text again.  There
are also subclasses for some common MIME object types, and a few
helper modules for extracting and parsing message field values.

The following sections describe the functionality of the
\module{mimelib} package.  The ordering follows a progression common
in applications: a MIME document is read as flat text from a file or
standard input, the text is parsed to produce an object model
representation of the MIME document, this model is manipulated and
queried, and finally the model is rendered back into flat text.  First
though, there's an introduction to the basic
\class{mimelib.Message} class.

It is perfectly feasible to create the object model out of whole cloth
-- i.e. completely from scratch.  From there, a similar progression can
be taken.  Also covered in the following sections are the exception
classes you might encounter while using \module{mimelib}, some
auxiliary utilities, and a few examples.

\section{The \class{mimelib.Message} class}

The \class{mimelib.Message} class (hereafter referred to simply as
\class{Message}) is the base class for the
\module{mimelib} object model.  It provides the core functionality for
setting and querying header fields, and for accessing message bodies.
It knows nothing about parsing or generating plain text documents.

Conceptually, a \class{Message} object consists of \emph{headers} and
\emph{payloads}.  Headers are \rfc{822} style field name and
values where the field name and value are separated by a colon.
Headers are stored and returned in case-preserving form but are
matched case-insensitively.  There may also be a single
\samp{Unix-From} header, also known as the envelope header or the
\code{From_} header.  The payload is either a string in the case of
simple message objects, or a list of \class{Message} objects for
multipart MIME documents.

\class{Message} objects provide a mapping style interface for
accessing the message headers, and an explicit interface for accessing
both the headers and the payload.  Here are the methods of the
\class{Message} class:

\begin{methoddesc}[Message]{ismultipart}{}
Return 1 if the payload is a list of sub-\class{Message} objects,
otherwise return 0 (implying that the payload is a simple string
object).
\end{methoddesc}

\begin{methoddesc}[Message]{set_unixfrom}{unixfrom}
Set the \samp{Unix-From} (a.k.a envelope header or \code{From_}
header) to \var{unixfrom}, which should be a string.
\end{methoddesc}

\begin{methoddesc}[Message]{get_unixfrom}{}
Return the \samp{Unix-From} header.  Defaults to \code{None} if the
\samp{Unix-From} header was never set.
\end{methoddesc}

\begin{methoddesc}[Message]{add_payload}{payload}
Add \var{payload} to the message object's existing payload.  If, prior
to calling this method, the object's payload was \code{None}
(i.e. never before set), then after this method is called, the payload
will be the argument \var{payload}.

If the object's payload was already a list
(i.e. \method{ismultipart()} returns 1), then \var{payload} is
appended to the end of the existing payload list.

For any other type of existing payload, \method{add_payload()} will
transform the new payload into a list consisting of the old payload
and \var{payload}, but only if the document is already a MIME
multipart document.  This condition is satisfied if the message's
\code{Content-Type:} header's main type is either \var{multipart}, or
there is no \code{Content-Type:} header.  In any other situation,
\exception{MultipartConversionError} is raised.
\end{methoddesc}

\begin{methoddesc}[Message]{get_payload}{\optional{i}}
Return the current payload, which will be a list of \class{Message}
objects when \method{ismultipart()} returns 1, or a scalar (hopefully
a string) when \method{ismultipart()} returns 0.

With optional \var{i}, \method{get_payload()} will return the
\var{i}-th element of the payload, if \method{ismultipart()} returns
1.  An \code{IndexError} will be raised if \var{i} is less than 0 or
greater than the number of items in the payload.  If the payload is
scalar (i.e. \method{ismultipart()} returns 0) and \var{i} is given, a
\code{TypeError} is raised.
\end{methoddesc}

\begin{methoddesc}[Message]{set_payload}{payload}
Set the entire message object's payload to \var{payload}.  It is the
client's responsibility to ensure the payload invariants.
\end{methoddesc}

The following methods implement a mapping-like interface for accessing
the message object's \rfc{822} headers.  Note that there are some
semantic differences between these methods and a normal mapping
(i.e. dictionary) interface.  For example, in a dictionary there are
no duplicate keys, but there may be duplicate message headers.  Also,
in dictionaries there is no guaranteed order to the keys returned by
\method{keys()}, but in a \class{Message} object, there is an explicit
order.  These semantic differences are intentional and are biased
toward maximal convenience.

Note that in all cases, any optional \samp{Unix-From} header the message
may have is not included in the mapping interface.

\begin{methoddesc}[Message]{__len__}{}
Return the total number of headers, including duplicates.
\end{methoddesc}

\begin{methoddesc}[Message]{__getitem__}{name}
Return the value of the named header field.  \var{name} should not
include the colon field separator.  If the header is missing,
\code{None} (a \code{KeyError} is never raised).

Note that if the named field appears more than once in the message's
headers, exactly which of those field values will be returned is
undefined.  Use the \method{getall()} method to get the values of all
the extant named headers.
\end{methoddesc}

\begin{methoddesc}[Message]{__setitem__}{name, val}
Add a header to the message with field name \var{name} and value
\var{val}.  The field is appended to the end of the message's headers.

Note that this does \emph{not} overwrite or delete any existing header
with the same name.  If you want to ensure that the new header is the
only one present in the message with field name
\var{name}, first use \method{__delitem__()} to delete all named
headers, e.g.:

\begin{verbatim}
del msg['subject']
msg['subject'] = 'Python roolz!'
\end{verbatim}
\end{methoddesc}

\begin{methoddesc}[Message]{__delitem__}{name}
Delete all occurrences of the field with name \var{name} from the
message's headers.  No exception is raised if the named field isn't
present in the headers.
\end{methoddesc}

\begin{methoddesc}[Message]{has_key}{name}
Return 1 if the message contains a header field named \var{name},
otherwise return 0.
\end{methoddesc}

\begin{methoddesc}[Message]{keys}{}
Return a list of all the message's header field names.  These keys
will be sorted in the order in which they were added to the message
via \method{__setitem__()}, and may contain duplicates.  Any fields
deleted and then subsequently re-added are always appended to the end
of the header list.
\end{methoddesc}

\begin{methoddesc}[Message]{values}{}
Return a list of all the message's field values.  These will be sorted
in the order in which they were added to the message via
\method{__setitem__()}, and may contain duplicates.  Any fields
deleted and then subsequently re-added are always appended to the end
of the header list.
\end{methoddesc}

\begin{methoddesc}[Message]{items}{}
Return a list of 2-tuples containing all the message's field headers and
values.  These will be sorted in the order in which they were added to
the message via \method{__setitem__()}, and may contain duplicates.
Any fields deleted and then subsequently re-added are always appended
to the end of the header list.
\end{methoddesc}

\begin{methoddesc}[Message]{get}{name\optional{, failobj}}
Return the value of the named header field.  This is identical to
\method{__getitem__()} except that optional \var{failobj} is returned
if the named header is missing (defaults to \code{None}).
\end{methoddesc}

Here are some additional useful methods:

\begin{methoddesc}[Message]{getall}{name\optional{, failobj}}
Return a list of all the values for the field named \var{name}.  These
will be sorted in the order in which they were added to the message
via \method{__setitem__()}.  Any fields
deleted and then subsequently re-added are always appended to the end
of the list.

If there are no such named headers in the message, \var{failobj} is
returned (defaults to \code{None}).
\end{methoddesc}

\begin{methoddesc}[Message]{addheader}{_name, _value, **_params}
Extended header setting.  This method is similar to
\method{__setitem__()} except that additional header parameters can be
provided as keyword arguments.  \var{_name} is the header to set and
\var{_value} is the ``primary'' value for the header.

For each item in the keyword argument dictionary \var{_params}, the
key is taken as the parameter name, with underscores converted to
dashes (since dashes are illegal in Python identifiers).  Normally,
the parameter will be added as \code{key="value"} unless the value is
\code{None}, in which case only the key will be added.

Here's an example:

\begin{verbatim}
msg.addheader('Content-Disposition', 'attachment', filename='bud.gif')
\end{verbatim}

This will add a header that looks like

\begin{verbatim}
Content-Disposition: attachment; filename="bud.gif"
\end{verbatim}
\end{methoddesc}

\begin{methoddesc}[Message]{gettype}{\optional{failobj}}
Return the message's content type, as a string of the form
``maintype/subtype'' as taken from the \code{Content-Type:} header.
The returned string is coerced to lowercase.

If there is no \code{Content-Type:} header in the message,
\var{failobj} is returned (defaults to \code{None}).
\end{methoddesc}

\begin{methoddesc}[Message]{getmaintype}{\optional{failobj}}
Return the message's main content type.  This essentially returns the
``maintype'' part of the string returned by \method{gettype()}, with the
same semantics for \var{failobj}.
\end{methoddesc}

\begin{methoddesc}[Message]{getsubtype}{\optional{failobj}}
Return the message's sub-content type.  This essentially returns the
``subtype'' part of the string returned by \method{gettype()}, with the
same semantics for \var{failobj}.
\end{methoddesc}

\begin{methoddesc}[Message]{getparams}{\optional{failobj}\optional{, header}}
Return the field parameters for the \code{Content-Type:} header as a
list of strings.  If the message has no \code{Content-Type:} header,
then \var{failobj} is returned (defaults to \code{None}).

Optional \var{header} if given, specifies the message header to use
instead of \code{Content-Type:}.
\end{methoddesc}

\begin{methoddesc}[Message]{getparam}{param\optional{,
    failobj}\optional{, header}}
Return the value of the \code{Content-Type:} header's parameter
\var{param} as a string.  If the message has no \code{Content-Type:}
header or if there is no such parameter, then \var{failobj} is
returned (defaults to \code{None}).

Optional \var{header} if given, specifies the message header to use
instead of \code{Content-Type:}.
\end{methoddesc}

\class{Message} objects can also optionally contain two instance
attributes, which can be used when generating the plain text of a MIME
message.

\begin{datadesc}{preamble}
The format of a MIME document allows for some text between the blank
line following the headers, and the first multipart boundary string.
Normally, this text is never visible in a MIME-aware mail reader
because it falls outside the standard MIME armor.  However, when
viewing the raw text of the message, or when viewing the message in a
non-MIME aware reader, this text can become visible.

The \var{preamble} attribute, if set, contains this leading
extra-armor text for MIME documents.  When the \class{Parser}
discovers some text after the headers but before the first boundary
string, it assigns this text to the message's \var{preamble}
attribute.  When the \class{Generator} is writing out the plain text
representation of a MIME message, and it finds the message has a
\var{preamble} attribute, it will write this text in the area between
the headers and the first boundary.

Note that if the message object has no preamble, there will be no
\var{preamble} attribute on the object.
\end{datadesc}

\begin{datadesc}{epilogue}
The \var{epilogue} attribute acts the same way as the \var{preamble}
attribute, except that it contains text that appears between the last
boundary and the end of the message.
\end{datadesc}

\section{Parsing MIME documents}
The \class{mimelib.Parser} class (hereafter referred to as
\class{Parser}) is used to take a message in flat text form and create
the associated object model.  The resulting object tree can then be
manipulated using the \class{Message} class interface and turned over
to a \class{Generator} to return the textual representation of the
message.  It is intended that the \class{Parser} to \class{Generator}
path be idempotent if the object model isn't modified in between.

\begin{classdesc}{Parser}{\optional{_class}}
The constructor for the \class{Parser} class takes a single optional
argument \var{_class}.  This must be callable factory (i.e. a function
or a class), and it is used whenever a sub-message object needs to be
created.   It defaults to \class{Message}.  \var{_class} will be
called with zero arguments.
\end{classdesc}

The other public \class{Parser} methods are:

\begin{methoddesc}[Parser]{parse}{fp}
Read all the data from the file-like object \var{fp}, parse the
resulting text, and return the root message object.  \var{fp} must
support both the \method{readline()} and the \method{read()} methods
on file-like objects.

The text contained in \var{fp} must be formatted as a block of \rfc{822}
style headers and header continuation lines, optionally preceeded by a
\samp{Unix-From} header.  The header block is terminated either by the
end of the data or by a blank line.  Following the header block is the
body of the message (which may contain MIME encoded sub-messages).
\end{methoddesc}

\begin{methoddesc}[Parser]{parsestr}{text}
Similar to the \method{parse()} method, except it takes a string
object instead of a file-like object.  Calling this method on a string
is exactly equivalent to wrapping \var{text} in a \class{StringIO}
instance first and calling \method{parse()}.
\end{methoddesc}

\section{Generating MIME documents}

The \class{mimelib.Generator} class (hereafter referred to as
\class{Generator}) is used to render a message object model into its
flat text representation, including MIME encoding any sub-messages,
generating the correct \rfc{822} headers, etc.

\begin{classdesc}{Generator}{outfp\optional{, mangle_from_}}
The constructor for the \class{Generator} class takes a file-like
object called \var{outfp} for an argument.  \var{outfp} must support
the \method{write()} method and be usable as the output file in a
Python 2.0 extended print statement.

Optional \var{mangle_from_} is a flag that, when true, puts a ``>''
character in front of any line in the body that starts exactly as
\samp{From } (i.e. \code{From} followed by a space at the front of the
line).  This is the only guaranteed portable way to avoid having such
lines be mistaken for \samp{Unix-From} headers (see
\url{http://home.netscape.com/eng/mozilla/2.0/relnotes/demo/content-length.html}
 for details).
\end{classdesc}

The other public \class{Generator} methods are:

\begin{methoddesc}[Generator]{write}{msg\optional{, unixfrom}}
Print the textual representation of the message object tree rooted at
\var{msg} to the output file specified when the \class{Generator}
instance was created.  Sub-objects are visited depth-first and the
resulting text will be properly MIME encoded.

Optional \var{unixfrom} is a flag that forces the printing of the
\samp{Unix-From} (a.k.a. envelope header or \code{From_} header)
delimiter before the first \rfc{822} header of the root message
object.  If the root object has no \samp{Unix-From} header, a standard
one is crafted.  Set this to 0 to inhibit the printing of the
\samp{Unix-From} delimiter (the default is to print the \samp{Unix-From}
header).

Note that for sub-objects, no \samp{Unix-From} header is ever printed.
\end{methoddesc}

As an alternative to using the \class{Generator} class, the
\module{mimelib} package provides a mix-in class called
\class{StringableMixin} which can be found in the
\module{mimelib.StringableMixin}
module.  This mix-in class provides two methods for getting the
representation of the message tree as plain text.  To use this class,
you must multiply derive from
\class{StringableMixin} and a class such as \class{Message} or
\class{BaseMIME}.

\begin{methoddesc}[StringableMixin]{get_text}{\optional{unixfrom}}
Return a string containing the plain text of the message tree.  This
is essentially a convenience wrapper around calling the
\class{Generator} class with a \class{StringIO} argument.  Optional
\var{unixfrom} is a flag saying whether the \samp{Unix-From} line should
be printed (by default, it is not).
\end{methoddesc}

\begin{methoddesc}[ReprMixin]{__str__}{}
A method for supporting \samp{str(msg)}.  This simply calls
\method{get_text()} passing true to the \var{unixfrom}
flag\footnote{Really, \method{__str__()} should return a string that
does not contain the \samp{Unix-From} while \method{__repr__()} would
return a string that does contain the \samp{Unix-From}.  However, it
turns out to be more convenient (currently) for \code{str(msg)} to not
include the \samp{Unix-From}.  Practicality beats Purity.}.
\end{methoddesc}

\section{Creating MIME and message objects from scratch}

Ordinarily, you get a message object tree by passing some text to a
\class{Parser} instance, which parses the text and returns the root of
the tree.  However you can also build a complete object tree from
scratch, or even individual \class{Message} objects, by hand.  In fact,
you can also take an existing tree and add new \class{Message}
objects, move them around, etc.  This makes a very convenient
interface for slicing-and-dicing MIME messages.

You can create a new object tree by creating \class{Message}
instances, adding payloads and all the appropriate headers manually.
For MIME messages though, \module{mimelib} provides some convenient
classes to make things easier.  Each of these classes should be
imported from a module with the same name as the class, from within
the \module{mimelib} package.  E.g.:

\begin{verbatim}
import mimelib.Image.Image
\end{verbatim}

or

\begin{verbatim}
from mimelib.Text import Text
\end{verbatim}

Here are the classes:

\begin{classdesc}{MIMEBase}{_major, _minor, **_params}
This is the base class for all the MIME-specific subclasses of
\class{Message}.  Ordinarily you won't create instances specifically
of \class{MIMEBase}, although you could.  \class{MIMEBase} is provided
primarily as a convenient base class for more specific MIME-aware
subclasses.

\var{_major} is the \code{Content-Type:} major type (e.g. ``text'' or
``image''), and \var{_minor} is the \code{Content-Type:} minor type
(e.g. ``plain'' or ``gif'').  \var{_params} is a parameter key/value
dictionary and is passed directly to \method{Message.addheader()}.

The \class{MIMEBase} class always adds a \code{Content-Type:} header
(based on \var{_major}, \var{_minor}, and \var{_params}), and a
\code{MIME-Version:} header (always set to ``1.0'').
\end{classdesc}

\begin{classdesc}{Image}{_imagedata\optional{, _minor\optional{,
    _encoding\optional{, **_params}}}}

A subclass of \class{MIMEBase}, the \class{Image} class is used to
create MIME message objects of major type ``image''.  \var{_imagedata}
is a string containing the raw image data.  If this data can be
decoded by the standard Python module \module{imghdr}, then the
subtype will be automatically included in the \code{Content-Type:}
header.  Otherwise you can explicitly specify the image subtype via
the \var{_minor} parameter.  If the minor type could not be guessed
and \var{_minor} was not given, then \code{TypeError} is raised.

Optional \var{_encoder} is a callable (i.e. function) which will
perform the actual encoding of the image data for transport.  The
callable takes one argument, which is the \class{Image} object
instance.  It should use \method{get_payload()} and
\method{set_payload()} to change the payload to encoded form.  It
should also add any
\code{Content-Transfer-Encoding:} or other headers to the message
object as necessary.  The default encoding is base64.

\var{_params} are passed straight through to the \class{MIMEBase}
constructor.
\end{classdesc}

\begin{classdesc}{Text}{_text\optional{, _minor\optional{, _charset}}}
A subclass of \class{MIMEBase}, the \class{Text} class is used to
create MIME objects of major type ``text''.  \var{_text} is the string
for the payload.  \var{_minor} is the minor type and defaults to
``plain''.  \var{_charset} is the character set of the text and is
passed as a parameter to the \class{MIMEBase} constructor; it defaults
to ``us-ascii''.  No guessing or encoding is performed on the text
data, but a newline is appended to \var{_text} if it doesn't already
end with a newline.
\end{classdesc}

\begin{classdesc}{RFC822}{_msg}
A subclass of \class{MIMEBase}, the \class{RFC822} class is used to
create MIME objects of type ``message/rfc822''.  \var{_msg} is used as
the payload, and must be an instance of class \class{Message} (or a
subclass thereof), otherwise a \exception{TypeError} is raised.
\end{classdesc}

\section{\module{mimelib} exception classes}

The following exception classes are defined in the
\module{mimelib.Errors} module:

\begin{excclassdesc}{MessageError}{}
This is the base class for all exceptions that the \module{mimelib}
package can raise.  It is derived from the standard
\exception{Exception} class and defines no additional methods.
\end{excclassdesc}

\begin{excclassdesc}{MessageParseError}{}
This is the base class for exceptions thrown by the \class{Parser}
class.  It is derived from \exception{MessageError}.
\end{excclassdesc}

\begin{excclassdesc}{HeaderParseError}{}
Raised under some error conditions when parsing the \rfc{822} headers of
a message, this class is derived from \exception{MessageParseError}.
It can be raised from the \method{Parser.parse()} or
\method{Parser.parsestr()} methods.

Situations where it can be raised include finding a \samp{Unix-From}
header after the first \rfc{822} header of the message, finding a
continuation line before the first \rfc{822} header is found, or finding
a line in the headers which is neither a header or a continuation
line.
\end{excclassdesc}

\begin{excclassdesc}{BoundaryError}{}
Raised under some error conditions when parsing the \rfc{822} headers of
a message, this class is derived from \exception{MessageParseError}.
It can be raised from the \method{Parser.parse()} or
\method{Parser.parsestr()} methods.

Situations where it can be raised include not being able to find the
starting or terminating boundary in a \code{multipart/*} message.
\end{excclassdesc}

\begin{excclassdesc}{MultipartConversionError}{}
Raised when a payload is added to a \class{Message} object using
\method{add_payload()}, but the payload is already a scalar and the
message's \code{Content-Type:} main type is not either ``multipart''
or missing.  \exception{MultipartConversionError} multiply inherits
from \exception{MessageError} and the built-in \exception{TypeError}.
\end{excclassdesc}

\section{Auxiliary utilities}
There are several useful utilities provided with the \module{mimelib}
package.

\begin{classdesc}{MsgReader}{rootmsg}
This class (imported from the \module{mimelib.MsgReader} module),
provides a file-like interface to a message object tree.  Actually, at
the moment it only provides a limited subset of file methods, but more
may be added in the future.  Its primary purpose is to enable
line-by-line iteration over the textual representation of the message
object tree, without having to flatten it with a \class{Generator}
first.

\var{rootmsg} is the root of the object tree to iterate line-by-line
over.

\begin{methoddesc}{readline}{}
Return the next line as would be read from the flattened text
representation of the message object tree.  Like ``standard''
\method{readline()}, this returns the empty string when the end of the
message object tree is reached.  Note that message headers are
\emph{not} included in the line-by-line iteration, only the payloads
and recursively, the payloads of the sub-objects in a depth-first
traversal.
\end{methoddesc}

\begin{methoddesc}{readlines}{\optional{sizehint}}
Returns all the lines of the message bodies as would be read from the
flattened text representation of the message object tree.  If the
optional \var{sizehint} argument is present, instead of reading all
the lines, whole lines totalling approximately sizehint bytes are
read.  This method is provided for compatibility with Python 2.1's
\module{xreadlines} module.
\end{methoddesc}
\end{classdesc}

The \module{mimelib.address} module exports some of the more useful
address parsing related functions from the \module{rfc822} module.
They are mostly provided for convenience.

\begin{funcdesc}{quote}{str}
Return a new string with backslashes in \var{str} replaced by two
backslashes and double quotes replaced by backslash-double quote.
\end{funcdesc}

\begin{funcdesc}{unquote}{str}
Return a new string which is an \emph{unquoted} version of \var{str}.
If \var{str} ends and begins with double quotes, they are stripped
off.  Likewise if \var{str} ends and begins with angle brackets, they
are stripped off.
\end{funcdesc}

\begin{funcdesc}{parseaddr}{address}
Parse address -- which should be the value of some address-containing
field such as \code{To:} or \code{Cc:} -- into its constituent
``realname'' and ``email address'' parts.  Returns a tuple of that
information, unless the parse fails, in which case a 2-tuple of
\code{(None, None)} is returned.
\end{funcdesc}

\begin{funcdesc}{dump_address_pair}{pair}
The inverse of \method{parseaddr()}, this takes a 2-tuple of the form
\code{(realname, email_address)} and returns the string value suitable
for a \code{To:} or \code{Cc:} header.  If the first element of
\var{pair} is false, then the second element is returned unmodified.
\end{funcdesc}

Module \module{address} also provides the following useful function:

\begin{funcdesc}{getaddresses}{fieldvalues}
This method returns a list of 2-tuples of the form returned by
\code{parseaddr()}.  \var{fieldvalues} is a sequence of header field
values as might be returned by \method{Message.getall()}.  Here's a
simple example that gets all the recipients of a message:

\begin{verbatim}
from mimelib.address import getaddresses

tos = msg.getall('to')
ccs = msg.getall('cc')
resent_tos = msg.getall('resent-to')
resent_ccs = msg.getall('resent-cc')
all_recipients = getaddresses(tos + ccs + resent_tos + resent_ccs)
\end{verbatim}
\end{funcdesc}

The \module{mimelib.date} module also exports some other useful date
related functions from the \module{rfc822} module.  They are mostly
provided for convenience.

\begin{funcdesc}{parsedate}{date}
Attempts to parse a date according to the rules in \rfc{822}.
however, some mailers don't follow that format as specified, so
\function{parsedate()} tries to guess correctly in such cases. 
\var{date} is a string containing an \rfc{822} date, such as 
\code{'Mon, 20 Nov 1995 19:12:08 -0500'}.  If it succeeds in parsing
the date, \function{parsedate()} returns a 9-tuple that can be passed
directly to \function{time.mktime()}; otherwise \code{None} will be
returned.  Note that fields 6, 7, and 8 of the result tuple are not
usable.
\end{funcdesc}

\begin{funcdesc}{parsedate_tz}{date}
Performs the same function as \function{parsedate()}, but returns
either \code{None} or a 10-tuple; the first 9 elements make up a tuple
that can be passed directly to \function{time.mktime()}, and the tenth
is the offset of the date's timezone from UTC (which is the official
term for Greenwich Mean Time).  (Note that the sign of the timezone
offset is the opposite of the sign of the \code{time.timezone}
variable for the same timezone; the latter variable follows the
\POSIX{} standard while this module follows \rfc{822}.)  If the input
string has no timezone, the last element of the tuple returned is
\code{None}.  Note that fields 6, 7, and 8 of the result tuple are not
usable.
\end{funcdesc}

\begin{funcdesc}{mktime_tz}{tuple}
Turn a 10-tuple as returned by \function{parsedate_tz()} into a UTC
timestamp.  It the timezone item in the tuple is \code{None}, assume
local time.  Minor deficiency: this first interprets the first 8
elements as a local time and then compensates for the timezone
difference; this may yield a slight error around daylight savings time
switch dates.  Not enough to worry about for common use.
\end{funcdesc}

\begin{funcdesc}{formatdate}{\optional{timeval}}
Returns the time formatted as per Internet standards \rfc{822}
and updated by \rfc{1123}.  If \var{timeval} is provided, then it
should be a floating point time value as expected by
\method{time.gmtime()}, otherwise the current time is used.
\end{funcdesc}

\section{Requirements, downloading, installing, and contact information}

\module{mimelib} has been tested with Python 2.0 and Python 2.1; there
are currently no plans to support earlier versions of Python.

\module{mimelib} development is current hosted at SourceForge, and the
project summary information is available at
\url{http://sourceforge.net/projects/mimelib/}.  This documentation is
available on-line at \url{http://mimelib.sourceforge.net/}.  The goal
of the \module{mimelib} project at SourceForge is to solidify the API,
and work towards a stable enough release to petition for inclusion in
the Python standard library.  In any event, \module{mimelib} will be a
required component for GNU Mailman version 2.1 (see
\url{http://www.list.org/}).

\module{mimelib} is easy to install via Python's standard
\emph{distutils} facility.  Simply unpack the tarball at your
shell\footnote{These instructions are admittedly Unix-centric.
\module{mimelib} itself should be portable to any system that Python
supports, and I welcome suggestions for unpacking and installing
instructions for other platforms}
using a command such as:

\begin{verbatim}
# tar zxvf mimelib-X.Y.tar.gz
# cd mimelib-X.Y
# python setup.py install
\end{verbatim}

where \code{X.Y} refers to the specific version number of the latest
\module{mimelib} release, e.g. \code{mimelib-0.3.tar.gz}.

\emph{Note that if your \code{tar} command doesn't accept the \code{z}
option, you'll need the GNU gzip/gunzip package, and can alternatively
try this:}

\begin{verbatim}
# gunzip -c mimelib-X.Y.tar.gz | tar xvf -
# cd mimelib-X.Y
# python setup.py install
\end{verbatim}

That's all there is to it!  You should now be able to do something
like the following to make sure that you can access \module{mimelib}
from Python:

\begin{verbatim}
# python
Python 2.1 (#1, Apr 17 2001, 23:30:09) 
[GCC egcs-2.91.66 19990314/Linux (egcs-1.1.2 release)] on linux2
Type "copyright", "credits" or "license" for more information.
>>> import mimelib
>>> mimelib.__version__
'0.3'
\end{verbatim}

Note that the actual version number printed may be different.

You can contact the author of \module{mimelib} via:

\begin{verbatim}
    Barry Warsaw
    barry@digicool.com
    Pythonlabs Team
    Digital Creations
    http://barry.wooz.org
\end{verbatim}

%\section{Examples}

%\emph{I'll fill in some more examples as I get more experience with
%the library.}

% remove this when adding to the Python standard library
\end{document}
